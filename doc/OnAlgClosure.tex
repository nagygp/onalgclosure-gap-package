% generated by GAPDoc2LaTeX from XML source (Frank Luebeck)
\documentclass[a4paper,11pt]{report}

\usepackage{a4wide}
\sloppy
\pagestyle{myheadings}
\usepackage{amssymb}
\usepackage[utf8]{inputenc}
\usepackage{makeidx}
\makeindex
\usepackage{color}
\definecolor{FireBrick}{rgb}{0.5812,0.0074,0.0083}
\definecolor{RoyalBlue}{rgb}{0.0236,0.0894,0.6179}
\definecolor{RoyalGreen}{rgb}{0.0236,0.6179,0.0894}
\definecolor{RoyalRed}{rgb}{0.6179,0.0236,0.0894}
\definecolor{LightBlue}{rgb}{0.8544,0.9511,1.0000}
\definecolor{Black}{rgb}{0.0,0.0,0.0}

\definecolor{linkColor}{rgb}{0.0,0.0,0.554}
\definecolor{citeColor}{rgb}{0.0,0.0,0.554}
\definecolor{fileColor}{rgb}{0.0,0.0,0.554}
\definecolor{urlColor}{rgb}{0.0,0.0,0.554}
\definecolor{promptColor}{rgb}{0.0,0.0,0.589}
\definecolor{brkpromptColor}{rgb}{0.589,0.0,0.0}
\definecolor{gapinputColor}{rgb}{0.589,0.0,0.0}
\definecolor{gapoutputColor}{rgb}{0.0,0.0,0.0}

%%  for a long time these were red and blue by default,
%%  now black, but keep variables to overwrite
\definecolor{FuncColor}{rgb}{0.0,0.0,0.0}
%% strange name because of pdflatex bug:
\definecolor{Chapter }{rgb}{0.0,0.0,0.0}
\definecolor{DarkOlive}{rgb}{0.1047,0.2412,0.0064}


\usepackage{fancyvrb}

\usepackage{mathptmx,helvet}
\usepackage[T1]{fontenc}
\usepackage{textcomp}


\usepackage[
            pdftex=true,
            bookmarks=true,        
            a4paper=true,
            pdftitle={Written with GAPDoc},
            pdfcreator={LaTeX with hyperref package / GAPDoc},
            colorlinks=true,
            backref=page,
            breaklinks=true,
            linkcolor=linkColor,
            citecolor=citeColor,
            filecolor=fileColor,
            urlcolor=urlColor,
            pdfpagemode={UseNone}, 
           ]{hyperref}

\newcommand{\maintitlesize}{\fontsize{50}{55}\selectfont}

% write page numbers to a .pnr log file for online help
\newwrite\pagenrlog
\immediate\openout\pagenrlog =\jobname.pnr
\immediate\write\pagenrlog{PAGENRS := [}
\newcommand{\logpage}[1]{\protect\write\pagenrlog{#1, \thepage,}}
%% were never documented, give conflicts with some additional packages

\newcommand{\GAP}{\textsf{GAP}}

%% nicer description environments, allows long labels
\usepackage{enumitem}
\setdescription{style=nextline}

%% depth of toc
\setcounter{tocdepth}{1}





%% command for ColorPrompt style examples
\newcommand{\gapprompt}[1]{\color{promptColor}{\bfseries #1}}
\newcommand{\gapbrkprompt}[1]{\color{brkpromptColor}{\bfseries #1}}
\newcommand{\gapinput}[1]{\color{gapinputColor}{#1}}


\begin{document}

\logpage{[ 0, 0, 0 ]}
\begin{titlepage}
\mbox{}\vfill

\begin{center}{\maintitlesize \textbf{ OnAlgClosure \mbox{}}}\\
\vfill

\hypersetup{pdftitle= OnAlgClosure }
\markright{\scriptsize \mbox{}\hfill  OnAlgClosure  \hfill\mbox{}}
{\Huge \textbf{ OnAlgClosure/Frobenius and projective linear action on objects of positive
characteristic \mbox{}}}\\
\vfill

{\Huge  0.1 \mbox{}}\\[1cm]
{ 07/07/2017 \mbox{}}\\[1cm]
\mbox{}\\[2cm]
{\Large \textbf{ G{\a'a}bor P. Nagy\\
    \mbox{}}}\\
\hypersetup{pdfauthor= G{\a'a}bor P. Nagy\\
    }
\end{center}\vfill

\mbox{}\\
{\mbox{}\\
\small \noindent \textbf{ G{\a'a}bor P. Nagy\\
    }  Email: \href{mailto://nagyg@math.u-szeged.hu} {\texttt{nagyg@math.u-szeged.hu}}\\
  Homepage: \href{http://www.math.u-szeged.hu/~nagyg} {\texttt{http://www.math.u-szeged.hu/\texttt{\symbol{126}}nagyg}}\\
  Address: \begin{minipage}[t]{8cm}\noindent
 Bolyai Institute of the University of Szeged\\
 Aradi v{\a'e}rtan{\a'u}k tere 1\\
 H-6720 Szeged (Hungary)\\
 \end{minipage}
}\\
\end{titlepage}

\newpage\setcounter{page}{2}
\newpage

\def\contentsname{Contents\logpage{[ 0, 0, 1 ]}}

\tableofcontents
\newpage

     
\chapter{\textcolor{Chapter }{Usage}}\label{Chapter_Usage}
\logpage{[ 1, 0, 0 ]}
\hyperdef{L}{X86A9B6F87E619FFF}{}
{
  
\section{\textcolor{Chapter }{Installation}}\label{Chapter_Usage_Section_Installation}
\logpage{[ 1, 1, 0 ]}
\hyperdef{L}{X8360C04082558A12}{}
{
  Download und unpack the file 

 \href{http://www.math.u-szeged.hu/~nagyg/OnAlgClosure-0.1.tar.gz} {\texttt{http://www.math.u-szeged.hu/\texttt{\symbol{126}}nagyg/OnAlgClosure-0.1.tar.gz}}

 into your pkg directory. You can load the package with the command

 \texttt{LoadPackage("OnAlgClosure");}

 }

 
\section{\textcolor{Chapter }{Functions for AC-Frobenius automorphism actions}}\label{Chapter_Usage_Section_Functions_for_AC-Frobenius_automorphism_actions}
\logpage{[ 1, 2, 0 ]}
\hyperdef{L}{X8365BFD187A21BF7}{}
{
  

\subsection{\textcolor{Chapter }{AC{\textunderscore}FrobeniusAutomorphism}}
\logpage{[ 1, 2, 1 ]}\nobreak
\hyperdef{L}{X807441667D5D76BB}{}
{\noindent\textcolor{FuncColor}{$\triangleright$\ \ \texttt{AC{\textunderscore}FrobeniusAutomorphism({\mdseries\slshape q})\index{ACFrobeniusAutomorphism@\texttt{AC{\textunderscore}}\-\texttt{Frobenius}\-\texttt{Automorphism}}
\label{ACFrobeniusAutomorphism}
}\hfill{\scriptsize (function)}}\\
\textbf{\indent Returns:\ }
an AC-Frobenius automorphism 



 Creates the Frobenius map $x->x^q$ for the prime power \mbox{\texttt{\mdseries\slshape q}} which operates on objects of characteristic $p$: vectors, matrices (also in \textsf{CVEC} representation), polynomials and rational functions. The argument may be the
finite field \texttt{GF(q)} as well. }

 Although algebraic closure is not defined in \textsf{GAP} one can say that AC-Frobenius automorphisms act on the algebraic closure of
the prime field \texttt{GF(p)}. 

An AC-Frobenius automorphism has infinite order. By default, no inverse of an
AC-Frobenius automorphism is defined. AC-Frobenius automorphisms are not
mapping as \textsf{GAP} objects. 

These are the main differences to the \textsf{GAP} command \texttt{FrobeniusAutomorphism}. 

It must be easy to install methods for the action of an AC-Frobenius
automorphism on new classes of objects. 
\begin{Verbatim}[commandchars=!@|,fontsize=\small,frame=single,label=Example]
  !gapprompt@gap>| !gapinput@fr:=AC_FrobeniusAutomorphism(9);|
  AC_FrobeniusAutomorphism(3^2)
  !gapprompt@gap>| !gapinput@Z(3)^fr;|
  Z(3)
  !gapprompt@gap>| !gapinput@Z(27)^fr;|
  Z(3^3)^9
  !gapprompt@gap>| !gapinput@v:=Random(GF(27)^5);|
  [ Z(3)^0, Z(3^3)^23, Z(3^3)^23, Z(3^3)^17, Z(3)^0 ]
  !gapprompt@gap>| !gapinput@v^fr;|
  [ Z(3)^0, Z(3^3)^25, Z(3^3)^25, Z(3^3)^23, Z(3)^0 ]
  !gapprompt@gap>| !gapinput@cv:=CVec(v);|
  <cvec over GF(3,3) of length 5>
  !gapprompt@gap>| !gapinput@cv^fr;|
  <cvec over GF(3,3) of length 5>
  !gapprompt@gap>| !gapinput@cv^fr=v^fr;|
  true
\end{Verbatim}
 
\begin{Verbatim}[commandchars=!@|,fontsize=\small,frame=single,label=Example]
  !gapprompt@gap>| !gapinput@fr:=AC_FrobeniusAutomorphism(7^2);|
  AC_FrobeniusAutomorphism(7^2)
  !gapprompt@gap>| !gapinput@x:=Indeterminate(GF(7),"x");;|
  !gapprompt@gap>| !gapinput@pol:=(x^3-Z(7))/(x^2-Z(7^3));|
  (x^3+Z(7)^4)/(x^2+Z(7^3)^172)
  !gapprompt@gap>| !gapinput@pol^fr;|
  (x^3+Z(7)^4)/(x^2+Z(7^3)^220)
  !gapprompt@gap>| !gapinput@pol=pol^(fr^3);|
  true
\end{Verbatim}
 AC-Frobenius automorphisms of the same characteristic can be multiplied and
share a unique multiplicative identity. 
\begin{Verbatim}[commandchars=!@|,fontsize=\small,frame=single,label=Example]
  !gapprompt@gap>| !gapinput@fr:=AC_FrobeniusAutomorphism(9);|
  AC_FrobeniusAutomorphism(3^2)
  !gapprompt@gap>| !gapinput@fr^2;|
  AC_FrobeniusAutomorphism(3^4)
  !gapprompt@gap>| !gapinput@One(fr);|
  AC_FrobeniusAutomorphism(3^0)
  !gapprompt@gap>| !gapinput@AC_FrobeniusAutomorphism(8)*AC_FrobeniusAutomorphism(16);|
  AC_FrobeniusAutomorphism(2^7)
  !gapprompt@gap>| !gapinput@One(last)=One(fr);|
  false
\end{Verbatim}
 

\subsection{\textcolor{Chapter }{AC{\textunderscore}FrobeniusAutomorphismOrbit}}
\logpage{[ 1, 2, 2 ]}\nobreak
\hyperdef{L}{X7ABD25CA7A40D677}{}
{\noindent\textcolor{FuncColor}{$\triangleright$\ \ \texttt{AC{\textunderscore}FrobeniusAutomorphismOrbit({\mdseries\slshape fr, obj})\index{ACFrobeniusAutomorphismOrbit@\texttt{AC{\textunderscore}}\-\texttt{Frobenius}\-\texttt{Automorphism}\-\texttt{Orbit}}
\label{ACFrobeniusAutomorphismOrbit}
}\hfill{\scriptsize (function)}}\\
\textbf{\indent Returns:\ }
the Frobenius orbit of the given object as list 



 The (i+1)th element of the Frobenius orbit is $obj^(fr^i)$. }

 
\begin{Verbatim}[commandchars=!@|,fontsize=\small,frame=single,label=Example]
  !gapprompt@gap>| !gapinput@fr:=AC_FrobeniusAutomorphism(7^2);|
  AC_FrobeniusAutomorphism(7^2)
  !gapprompt@gap>| !gapinput@m:=[[0*Z(7),Z(7^3)],[Z(7^4)^-1,Z(7)^0]];|
  [ [ 0*Z(7), Z(7^3) ], [ Z(7^4)^2399, Z(7)^0 ] ]
  !gapprompt@gap>| !gapinput@AC_FrobeniusAutomorphismOrbit(fr,m);|
  [ [ [ 0*Z(7), Z(7^3) ], [ Z(7^4)^2399, Z(7)^0 ] ], [ [ 0*Z(7), Z(7^3)^49 ], [ Z(7^4)^2351, Z(7)^0 ] ], 
    [ [ 0*Z(7), Z(7^3)^7 ], [ Z(7^4)^2399, Z(7)^0 ] ], [ [ 0*Z(7), Z(7^3) ], [ Z(7^4)^2351, Z(7)^0 ] ], 
    [ [ 0*Z(7), Z(7^3)^49 ], [ Z(7^4)^2399, Z(7)^0 ] ], [ [ 0*Z(7), Z(7^3)^7 ], [ Z(7^4)^2351, Z(7)^0 ] ] ]
\end{Verbatim}
 An AC-Frobenius automorphism has infinite order. By default, no inverse of an
AC-Frobenius automorphism is defined. AC-Frobenius automorphisms are not
mapping as \textsf{GAP} objects. If you want, they act on the algebraic closure of the prime field \texttt{GF(p)}. In fact, these are the main differences to the \textsf{GAP} command \texttt{FrobeniusAutomorphism}. 

It must be easy to install methods for the action of an AC-Frobenius
automorphism on new classes of objects. }

 
\section{\textcolor{Chapter }{Functions for AC-projective linear transformations}}\label{Chapter_Usage_Section_Functions_for_AC-projective_linear_transformations}
\logpage{[ 1, 3, 0 ]}
\hyperdef{L}{X809ED78B82924069}{}
{
  

\subsection{\textcolor{Chapter }{AC{\textunderscore}ProjectiveLinearTransformation}}
\logpage{[ 1, 3, 1 ]}\nobreak
\hyperdef{L}{X7CD817EA7F9B1EA3}{}
{\noindent\textcolor{FuncColor}{$\triangleright$\ \ \texttt{AC{\textunderscore}ProjectiveLinearTransformation({\mdseries\slshape M})\index{ACProjectiveLinearTransformation@\texttt{AC{\textunderscore}}\-\texttt{Projective}\-\texttt{Linear}\-\texttt{Transformation}}
\label{ACProjectiveLinearTransformation}
}\hfill{\scriptsize (function)}}\\
\textbf{\indent Returns:\ }
an AC-projective linear transformation 



 Creates the projective linear transformation $\varphi$ corresponding to the $n\times n$ matrix \mbox{\texttt{\mdseries\slshape M}}. Here, \mbox{\texttt{\mdseries\slshape M}} is a nonsingular matrix over the finite field \texttt{GF(q)}. By definition $\varphi$ acts via \texttt{OnLines} on row vectors of length $n$ with entries from the algebraic closure of \texttt{GF(q)}. }

 
\begin{Verbatim}[commandchars=!@|,fontsize=\small,frame=single,label=Example]
  !gapprompt@gap>| !gapinput@m1:=[[ Z(5^2), 0*Z(5), 0*Z(5) ],[ 0*Z(5), Z(5)^0, 0*Z(5) ],[ 0*Z(5), 0*Z(5), Z(5^2)^19 ]];;|
  !gapprompt@gap>| !gapinput@m2:=[[ Z(5^2)^13, Z(5)^2, Z(5)^0 ],[ Z(5)^2, Z(5)^2, 0*Z(5) ],[ Z(5)^0, 0*Z(5), 0*Z(5) ]];;|
  !gapprompt@gap>| !gapinput@t1:=AC_ProjectiveLinearTransformation(m1);|
  AC_ProjectiveLinearTransformation([ [ Z(5)^0, 0*Z(5), 0*Z(5) ], [ 0*Z(5), Z(5^2)^23, 0*Z(5) ], [ 0*Z(5), 0*Z(5), Z(5)^3 ] ])
  !gapprompt@gap>| !gapinput@t2:=AC_ProjectiveLinearTransformation(m2);|
  AC_ProjectiveLinearTransformation([ [ Z(5)^0, Z(5^2)^23, Z(5^2)^11 ], [ Z(5^2)^23, Z(5^2)^23, 0*Z(5) ], 
    [ Z(5^2)^11, 0*Z(5), 0*Z(5) ] ])
  !gapprompt@gap>| !gapinput@Order(t1);|
  24
  !gapprompt@gap>| !gapinput@t1*t2;|
  AC_ProjectiveLinearTransformation([ [ Z(5)^0, Z(5^2)^23, Z(5^2)^11 ], [ Z(5^2)^22, Z(5^2)^22, 0*Z(5) ], 
    [ Z(5^2)^5, 0*Z(5), 0*Z(5) ] ])
  !gapprompt@gap>| !gapinput@t1/t2;|
  AC_ProjectiveLinearTransformation([ [ 0*Z(5), 0*Z(5), Z(5)^0 ], [ 0*Z(5), Z(5^2)^11, Z(5^2)^11 ], [ Z(5)^3, Z(5), Z(5^2)^11 ] ])
  !gapprompt@gap>| !gapinput@One(t1);|
  AC_ProjectiveLinearTransformation([ [ Z(5)^0, 0*Z(5), 0*Z(5) ], [ 0*Z(5), Z(5)^0, 0*Z(5) ], [ 0*Z(5), 0*Z(5), Z(5)^0 ] ])
  !gapprompt@gap>| !gapinput@Characteristic(t1);|
  5
\end{Verbatim}
 AC-projective linear transformations of the same characteristic and dimension
can be multiplied, possess an inverse AC-projective linear transformation and
share a unique multiplicative identity. 

AC-proejctive linear transformations defined over \texttt{GF(q)} have a regular permutation action on $PG(n-1,q)$. Via nice monomorphism, $\varphi$ knows this permutation. This enables efficient arithmetics for groups
generated by AC-projective linear transformations. 

It must be easy to implement other actions of AC-projective linear
transformations on GAP objects. 
\begin{Verbatim}[commandchars=!@|,fontsize=\small,frame=single,label=Example]
  !gapprompt@gap>| !gapinput@q:=5;|
  5
  !gapprompt@gap>| !gapinput@mg:=GU(3,q);|
  GU(3,5)
  !gapprompt@gap>| !gapinput@Size(mg);|
  2268000
  !gapprompt@gap>| !gapinput@oset:=Orbit(mg,Z(q)^0*[0,0,1],OnLines);; |
  !gapprompt@gap>| !gapinput@Size(oset); q^3+1;|
  126
  126
  !gapprompt@gap>| !gapinput@pg:=AC_ProjectiveTransformationGroupWithShortOrbit(mg,oset);|
  <group with 2 generators>
  !gapprompt@gap>| !gapinput@Size(pg); Size(PGU(3,q));|
  378000
  378000
  !gapprompt@gap>| !gapinput@vec:=Z(q)*[1,Z(q^3),0];|
  [ Z(5), Z(5^3)^32, 0*Z(5) ]
  !gapprompt@gap>| !gapinput@OrbitLength(pg,NormedRowVector(vec));|
  75600
  !gapprompt@gap>| !gapinput@StructureDescription(pg);|
  "PSU(3,5) : C3"
\end{Verbatim}
 

\subsection{\textcolor{Chapter }{AC{\textunderscore}ProjectiveTransformationGroupWithShortOrbit}}
\logpage{[ 1, 3, 2 ]}\nobreak
\hyperdef{L}{X870D421C7BB5E273}{}
{\noindent\textcolor{FuncColor}{$\triangleright$\ \ \texttt{AC{\textunderscore}ProjectiveTransformationGroupWithShortOrbit({\mdseries\slshape matgr, orb})\index{ACProjectiveTransformationGroupWithShortOrbit@\texttt{AC{\textunderscore}}\-\texttt{Projective}\-\texttt{Transformation}\-\texttt{Group}\-\texttt{With}\-\texttt{Short}\-\texttt{Orbit}}
\label{ACProjectiveTransformationGroupWithShortOrbit}
}\hfill{\scriptsize (function)}}\\
\textbf{\indent Returns:\ }
a projective linear cycle 



 Creates the AC-projective linear transformation group $G$ corresponding to the matrix group \mbox{\texttt{\mdseries\slshape matgr}}. \texttt{matgr} must have a faithful action on the orbit \mbox{\texttt{\mdseries\slshape orb}}. This permutation action is stored as nice monomorphism of $G$. }

 }

 }

 \def\indexname{Index\logpage{[ "Ind", 0, 0 ]}
\hyperdef{L}{X83A0356F839C696F}{}
}

\cleardoublepage
\phantomsection
\addcontentsline{toc}{chapter}{Index}


\printindex

\immediate\write\pagenrlog{["Ind", 0, 0], \arabic{page},}
\newpage
\immediate\write\pagenrlog{["End"], \arabic{page}];}
\immediate\closeout\pagenrlog
\end{document}
